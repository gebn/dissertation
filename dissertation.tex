\documentclass[12pt,a4]{bhamdissertation}

%load any additional packages
\usepackage[utf8]{inputenc}
\usepackage{amssymb}
\usepackage[UKenglish]{babel}
\usepackage[autostyle]{csquotes}
\usepackage[nodayofweek]{datetime}
\usepackage{pdflscape}
\usepackage{titlesec}
\usepackage[backend=biber,bibencoding=utf8,style=ieee,sorting=none]{biblatex}
\usepackage[hidelinks]{hyperref}
\usepackage[toc,page]{appendix}
\usepackage{lipsum}
\usepackage{enumitem}

\titleformat{\paragraph}{\normalfont\normalsize\bfseries}{\theparagraph}{1em}{}
\titlespacing*{\paragraph}{0pt}{3.25ex plus 1ex minus .2ex}{1.5ex plus .2ex}

%input macros (i.e. write your own macros file called mymacros.tex 
%and uncomment the next line)
%\include{mymacros}

\title{Equaliser}
\author{George Brighton}             %your name
\college{School of Computer Science}  %your college
\degree{MEng Computer Science \& Software Engineering}     %the degree

\addbibresource{references.bib}
\bibliography{references}        %use a bibtex bibliography file refs.bib
%\bibliographystyle{plain}  %use the plain bibliography style

%end the preamble and start the document
\begin{document}

%this baselineskip gives sufficient line spacing for an examiner to easily
%markup the thesis with comments
\baselineskip=18pt plus1pt

%set the number of sectioning levels that get number and appear in the contents
\setcounter{secnumdepth}{3}
\setcounter{tocdepth}{1}

\maketitle                  % create a title page from the preamble info

\begin{romanpages}          % start roman page numbering

\tableofcontents

% TODO should be possible to move contents line to class file
% http://tex.stackexchange.com/a/258973 for why 3 things are needed
\cleardoublepage\phantomsection\addcontentsline{toc}{chapter}{Abstract}
\begin{abstract}
\lipsum[1-1] % TODO
\end{abstract}

\cleardoublepage\phantomsection\addcontentsline{toc}{chapter}{Acknowledgements}
\begin{acknowledgements}
\lipsum[2-2] % TODO
\end{acknowledgements}

\end{romanpages}            % end roman page numbering

\chapter{Introduction}
% TODO

\chapter{Mining properties of existing systems}

The insights in this section are based on a whitepaper \autocite{B17} I produced earlier in the project on the current state of the ticketing industry. It is clear that there is currently no complete solution to the problem of touting. This section qualitatively analyses the research in the paper to identify attributes that are conducive and resistant to touting, and other notable features whose absence or presence would enhance a future ticket distribution system.

\section{Single Distributor}

Event organisers using multiple distributors is clearly an issue, as it allows any protections offered by a system to be circumvented by a vulnerable one. Malicious actors would simply use the latter. DICE identified this issue, and plugged the hole by making themselves the exclusive distributor of tickets for a given event. If two systems existed in future that were secure in isolation, using them together could introduce potential issues or opportunities for confusion, so in practice, one distributor should be used.

Box offices are inherently at odds with this observation on several levels. If a venue's box office is to be its sole distributor of tickets, using a separate system for each venue would be a great inconvenience to fans and event organisers. If we make the reasonable assumption that a box office alone cannot handle the demand for tickets, then another distributor is needed, which means the box office must be removed to maintain the single source. Unfortunately, from a touting prevention perspective, box offices are pointless at best, and a subversive side-channel attack at worst.

\subsection{Conflict of Interest}

Given the rule of a sole distributor, companies will compete to win the business of being that distributor. To differentiate themselves, these firms can highlight the rate at which they sell tickets, and the total number of tickets sold, which may conflict with ensuring those tickets go to genuine fans. This can only be solved by removing all metrics related to the rate of ticket sales from a distributor's fee calculation. They should receive a fixed payment that is agreed in advance and independent of the number of tickets sold. Promotion must be distinct from processing.

\section{Ticket Transferability}

Paper tickets are inherently broken, as it is impossible to prevent a physical object from being transferred between two parties. Adding a name which is rarely validated does nothing to improve this situation. Paperless tickets were ostensibly introduced to combat this. The photo ID was an important improvement, however as already discussed, only the purchaser's identity is validated (if checked at all), leaving 3-7 tickets per credit card available for touting, and a tidy profit. Unfortunately, DICE is ineffective against touts in this regard. If an individual buys several tickets, there is nothing tying these to individuals, and so they can be resold.

Automation of the ticket purchasing process attacks the same vulnerability, and the reCAPTCHA test used by Ticketmaster and others is duct tape over a deeper problem. A system that is vulnerable to bots is also vulnerable to humans, if to a lesser extent. Limiting the number of tickets that can be purchased per transaction is another smoking gun, and has only resulted in touts using multiple credit cards. The only way to truly fix these two issues is to address the transferability problem.

The entire reselling industry only exists because tickets are, in practice, transferable. Twickets is well-intentioned, but does not address the root of the problem. It should not be possible for tickets to go near a secondary ticketing site, as touts are under no obligation to use a benevolent one. Therefore, the solution of for an identity to be tied to a ticket, and validated on entry to the venue. The validation is key; no amount of security will help if it is looking the other way.

\subsection{Reallocation}

Ticket reallocation occurs when an individual with a ticket wants to return it for a refund. DICE made the key design decision to not allow the purchaser any say in who returned tickets go to. By simply returning tickets to a pool, a tout cannot resell a ticket, as they have no control over who receives it. In a system that identified every ticketholder, this would be a useful property, however unfortunately, due to the aforementioned issue with only validating the purchaser's identity, it is redundant for DICE. As tickets are nameless, a tout can replace those he lets in as many times as he likes before the event. 

\section{Notable Features}

In addition to properties directly related to the prevention of touting, it is useful to note ways in which these systems are either suboptimal or successful.

\subsection{Purchase Timeout}

The concept of entering a minimal set of details, then having a fixed amount of time to complete the purchase is very useful. It levels the playing field, meaning people are not pelanised for being slow at typing or mis-entering payment information. It also reduces the amount of time users are under pressure for, without hindering competition, leading to a more pleasant ordering experience.

\subsection{Stability}

The inability for ticket websites to stay available during large ticket releases appears to have become something of a requirement. Since the introduction of cloud computing several years ago, there has been no excuse for being unable to handle spikes in demand. Any ticket vendor that meets the seemingly straightforward criterion of staying online during a large release would already be ahead of the competition.

\subsection{Group Entrance}

Due to the way paperless tickets work, all members of a group must enter the venue at once. While not broken entirely, this can cause serious inconvenience if one person in a large group is running late. A secondary issue here is that of gifted tickets. As the payment card is effectively the ticket, it is not possible to buy tickets for two friends to attend an event on their own. Each person in a group should be able to enter the venue in their own right. This should not be a challenging requirement if each ticket is individually tied to a person.

\subsection{Fees}

Fees are one of the most disliked properties of existing systems, yet one of the easiest to fix. It is clear that current fees are not related to their label, the most obvious example being Ticketmaster charging for delivery of tickets sent via email. Customers know they are used to maximise revenue while keeping the perceived price of tickets as low as possible.

Removing these fees is obviously the ideal, but a more pragmatic solution would be to include them in the ticket price, so buyers see a single value. This would be more popular with customers, as it is more honest. The increase in face value will not put people off or leave a sour taste nearly as much as a list of questionable charges. DICE got this right by showing no fees to purchasers, and buyer perception is amongst the highest of any distributor.

\chapter{Requirements}

Analysis of the current state of the art with regard to ticket touting provided a ruler by which to measure any new ticket system. By flipping this on its head, I hoped to use these metrics to design core requirements for a perfect system. Many more requirements would emerge during the practical design, however they were not perceived in advance, simply because no system so far was able to offer a complete solution to ticket touting.

\section{Functional}

\begin{enumerate}
    \item Users should be able to return a ticket for a refund.
    \begin{enumerate}
        \item This should be permitted outright, or only if the system can reallocate those tickets to someone in the waiting list, so no revenue is lost.
    \end{enumerate}
    \item Direct ticket transfers between individuals should not be permitted. User A should not be able to give their ticket to User B, as we cannot prevent money from changing hands outside of the application.
    \begin{enumerate}
        \item Unwanted or refunded tickets should be returned to a pool of available tickets.
    \end{enumerate}
    \item Users should only have to enter a minimum number of details before their tickets are reserved, or they enter the waiting list.
    \item Venue staff must be able to revalidate a ticket and verify an attendee's identity inside the venue.
    \item Two-factor authentication should be mandatory for login.
\end{enumerate}

\subsection{Groups}

\begin{enumerate}[resume]
    \item Each ticket in a group booking must be tied to a registered user at the point of order.
    \item Each attendee in a group should be able to enter the venue individually on their own merit.
    \item A group leader must be able to specify that some members in a group should pay for their own tickets.
\end{enumerate}

\subsection{Gifts}

\begin{enumerate}[resume]
    \item A user should be able to purchase tickets for others.
    \item They should be able to individually specify when those tickets should become visible to the giftees.
    \item The gifter should be able to attach a message to each gifted ticket that will be displayed when it is revealed to the giftee.
    \item The gifter can refund a ticket before it is released to the giftee, after which only the giftee can refund it.
    \item The original purchaser is always refunded the cost of the ticket, even if a giftee initiates a refund.
\end{enumerate}

\section{Non-functional}

\begin{enumerate}[resume]
    \item The system must partner with event organisers to be the one and only distributor of tickets. Only tickets issued by the system and verified by staff upon entry should be accepted.
    \item Fees should be included in the cost of the ticket.
    \item The ordering system should be able to support 500,000 simultaneous users.
    \item No subsystem should have a single point of failure.
\end{enumerate}

\chapter{Practical Design}

My task was now to design a system, using the requirements in the previous chapter as non-negotiables. Even with a set of requirements, it is difficult to know where to start designing a system. Ticket distribution is not particularly large, but the human aspect meant there was nowhere to hide: if the design was flawed, it would be no better than any other solution on the market.

\section{Entrance}

I decided to start by focusing on the process by which ticketholders would gain access to a venue, as controlling this is ultimately the main purpose of the system. If at all possible, I wanted this to be completely automatic, with no human verification needed. This would remove any chance of human error or corruption, while making my solution more attractive to venues, who would not have to hire as many staff.

To make this work, I knew I would have to produce a token that could only be used by its rightful owner to enter a venue. At first, I considered payment cards, for the reason that many experts have already put a lot of effort into making these secure. If possible, I wanted to stand on their shoulders. Unfortunately, this had two key problems.

\begin{enumerate}
    \item In a group scenario, only the original purchaser necessarily has a payment card on file. As the system would not know about the cards of anyone else in the booking, they would have to enter the venue together, which was a direct violation of the requirements.
    \item Requiring each individual to have a payment card on file is too draconian. Not everyone going to an Ed Sheeran concert is old enough to own a credit card, and it is unreasonable to require parents to give them to their children - and even then, there is only one copy of the card.
\end{enumerate}

This was clearly unworkable. By going through this process, I knew this token had to be something that everyone would have, preferably something they already had on them anyway for convenience. The most obvious choice was a mobile phone, which has the added benefit that I could put arbitrary data on it to act as a ticket. It is trivial to send this data via the internet, and as phones have a screen, they can display this data however I liked, which could be the same as existing ticket barcodes so no changes to barriers at venues would be required.

I quickly thought of a handful of ways that a barcode could be displayed on a phone securely. It could be a hash of some key material unique to the user, such as an X.509 private key. An intermediate certificate authority could be created for each event, that would sign requests for tickets. If a refund was initiated, the ticket could be revoked and so on. To make tickets difficult to copy, the key material could be hashed with a timestamp every 30 seconds, and an animation could be displayed to make it difficult to screenshot. This was not perfect, but I was reasonably confident of being able to tie a ticket to a phone.

Unfortunately, it was all for nothing. It is one thing to tie a ticket to a phone, but it is quite another to ensure the person who shows up with a phone is its true owner. Any number of security features is irrelevant if a tout can simply give a phone containing a ticket to someone else. Indeed, touting is so lucrative that one could quite easily buy 10 or 100 pay-as-you-go phones and include the cost of the phone in the price of the ticket, and still make a tidy profit. This problem turned out to be much harder than initially thought.

At first, I considered forcing the user to enter a passcode, however I realised the challenge cannot be based around knowledge, as that can be communicated to someone else. Similar to the ideas behind two-factor authentication, it requires something you have, not something you know. I then looked into biometrics, for example using a fingerprint to reveal the barcode for a few seconds at the barrier. This had four issues.

\begin{enumerate}
    \item Although becoming more prevalent, many phones are not equipped with fingerprint readers. Relying on them would require a two-lane system where those with less-able phones would queue. Any malicious actors would simply use the weaker option.
    \item Even amongst phones with the technology, they do not always function correctly. A 1\% mis-read rate equates to 900 people in Wembley Stadium.
    \item The phone is literally in the hands of the attacker, and biometrics fundamentally trust the phone's technology. Ideally, we only want to have to trust what is under our control.
    \item On iOS, applications receive a boolean response to a fingerprint scan. They cannot identify who a fingerprint belongs to, so an attacker could simply register a fingerprint of whoever bought a touted ticket.
\end{enumerate}

I decided the technology was neither secure nor reliable enough to use. I also considered photographs, which could be recognised at the barrier, but again the technology is not accurate enough.

Disheartened that phones alone did not seem to provide a secure option, I changed to using it in combination with another piece of identity. I briefly revisted credit cards as a possible option, particularly because they are not something given to a stranger lightly, but at the same time touts are known to own many cards, and could make buyers pay a deposit for the duration of an event. Passports seemed to be more promising, however they are almost too valuable. I found it difficult to come up with a document that was private enough to be worth validating, but not so private as to worry genuine purchasers. Thefts are common at concerts, and would be even more so if thieves knew everyone in attentance had a credit card, smartphone and passport on them.

Photocards and Oyster cards initially seemed to be an answer. They are difficult enough to get hold of and a pain to replace, however they are bespoke to each country (in the case of Oyster, a single city), and sometimes genuine copies look fake. The real issue was they are trivially easy to photocopy, and many countries, such as Germany, do not allow non-govermental organisations to use national identity cards [CITE]. Any automated system is vulnerable to forgeries. Without a human present, to check each document, we cannot really do anything about this.

\subsection{Manual Verification}

Regretfully, I resigned to having humans present at the turnstiles to verify attendees. Passport gates at the UK border still require a human to confirm access. If even these are not entirely automatic, when in complete control of the physical token (the passport), the chances are I would not be able to come up with a secure solution either. This was a disappointment, but no worse than the current paperless system.

To reduce the likelihood of staff intimidation, it could be done in a similar fashion to passport gates, where staff are kept physically separate from the people they are allowing through. To limit potential issues of corruption, where a member of staff lets their friends through, two controllers could manage several lanes, with individual requests for entry being randomised between them, so there was no guarantee of being confirmed by a single person. Regardless, this has the benefit of not requiring one employee per lane into the venue, so the increase in cost is not as substantial as I thought.

My attention then turned back to the exact checks to perform. Photos were showing the most promise, and humans are good at comparing faces, so I revisited this idea. I envisaged a system where users would upload a photograph of themself during registration, which would be shown alongside the barcode. A human would then ensure the person in front of them matched, and allow them through.

Photos would have to be vetted to make sure they were appropriate, however this could be done automatically to some extent. Facial recognition could be used to verify that one and only one face appeared in the image, and it was of appropriate size. Ultimately, such a mechanism may need a human to confirm, however this would only have to be done once per person. Of course, the photo would have to be verified before any tickets were allocated, and could not be changed at will. Any updates would have to be approved to ensure the person in the photo was the same. Ticketmaster has a similar verification system for the case where the payment card expires before an event.

I was happy with all of this, apart from the photo existing on the user's phone, which felt dangerous; as already mentioned, the phone is in the hands of Mallory. To remove this dependency, instead of showing the photo on the phone, when the barcode was scanned, the system would show the photograph on file next to the attendee, then the member of staff would simply accept or reject entry. This had the nice property that all ticket material on the phone need no longer be secure, as even armed with another person's ticket, it would not be possible to pass the visual check (identical twins notwithstanding, but they are unlikely to tout to each other). Not relying on any phone security is a nice assumption to be able to make. There was no longer any value in screenshotting someone else's phone, so I would not have to put lots of effort into trying to prevent this.

The outcome was realising the phone is effectively reduced to a pointer to a person, like a user ID. To remove the possibility of attackers iterating through all users, finding people who looked similar to each other, the token could be long and random, so it is infeasible to explore the space of all users.

Therefore, in summary, when a user registers, they upload a clear photo of themselves. They have a single ``ticket'' for all events, which is unique to them. Upon scanning this data at the ticket barrier, their profile is retrieved, and automatically checked to see whether they have a valid ticket for the event. If they do, a member of staff compares the photo on file with the person who has shown up. If they match, the person is allowed in. A given person is only allowed in once. As tickets in themselves are useless for entry, they cannot be touted. Even changing the barcode is unlikely to point to another user, and even if it does collide, the chances are astronomically small that they will look like the attacker, or have a ticket. Tokens could also be regenerated regularly completely automatically, without users noticing.

\section{The Event Hierarchy}

At this stage, I started confusing myself. ``event'' could refer to ``Wimbledon'', ``Wimbledon 2017'', ``Wimbledon 2017, Day 1'', or ``Wimbledon 2017, Day 1, Centre Court''. Some more accurate vocabulary was needed to avoid ambiguity.

\begin{description}[style=nextline]
    \item[Tiers] A type of ticket for a single performance, e.g. \textit{Standing}, \textit{Stalls}, or \textit{Debentures}. A tier has an associated price, availability and returns policy (always, or only if the ticket can be reallocated). It may or may not have a seat allocation.
    \item[Fixtures] Generally, a fixture will match the duration of a ticket's validity, e.g. \textit{UKNOF 38} or \textit{Wimbledon 2017, Day 4, Centre Court}. Fixtures have a designated venue, and one or more tiers.
    \item[Series] Fixtures combine to form series, which are collections of experiences, usually over no more than a couple of months. Examples include \textit{Glastonbury 2017}, \textit{Derren Brown: Svengali}, and \textit{Ed Sheeran: X Tour}. Their start and end dates can be derived from the fixtures within them.
    \item[Headlines] I considered this final tier to contain series, however they were ultimately not implemented, in favour of tags instead. Examples would have been \textit{Wimbledon} and \textit{6 Nations}.
\end{description}

I wanted the system to be able to handle as many types of event as possible, so generic terminology was needed. However, it need not be exhaustive. If a new type of event that does not fit the above structure came along, I decided it could adapt and, if necessary, abuse it to fix its needs. Ticketmaster is a frequent culprit in this regard.

\section{Etymology}

I was now beginning to see the system's role more clearly, and decided to call it \textit{Equaliser}. This had the nice property of applying to many different events, similar to the system, e.g. a mixing equaliser in the context of music, or an equaliser in a tennis match. I also saw it as a way f getting equal with touts.

\section{Ordering}

I now had some requirements for the registration process, and a robust way to ensure only those with valid tickets could enter a venue. Now I needed to work out how to join these two together, i.e. how registered users could obtain tickets.

\subsection{Individuals}

I decided to tackle the case of an individual attending a fixture alone first, as it does not involve the co-ordination required for groups. The initial stages of this practically designed themselves:

\begin{enumerate}
    \item The user chooses the series, fixture and tier they are interested in.
    \item The user submits this order.
    \item If a ticket is not available for that tier, the user enters the queue and waits until it is available.
    \item If and when the ticket becomes available, the user is alerted and has a set amount of time to pay; if payment is received within this time, the ticket is delivered to their phone, otherwise their order is cancelled.
\end{enumerate}

I decided to allow up to 30 minutes between 9am and 9pm for payment, with the potential for a shorter window as the event gets closer or the waiting list longer. If a ticket becomes available outside these times, it expire at 9:30am. If a ticket expires, the user could be sent to the back of the queue, but there is no guarantee that they will not miss it again, so rather than have them cycle through the queue, it is clearer to cancel their order. Anyone who truly wants to attend would not miss their chance to pay for their ticket.

\subsection{Fees}

Transactions made through the online shop will be paid into Equaliser's bank account. Event organisers will receive their majority share of that as agreed, with the venue also receiving a cut. It is no hardship to negotiate these fees, as the venue, event organiser and Equaliser already have to partner to prevent other ticket distributors allowing Equaliser to be bypassed.

\subsection{Waiting List}

The fact that a ticket may not be available immediately complicates things, but Equaliser would be fairly useless if it could not handle this situation. Tickets may become available either through refunds, or if ticket release is staggered. Intuitively, as tiers can sell out individually, I envisaged one waiting list per tier, however after deliberation, I found some subtle problems.

\begin{enumerate}
    \item A user may not care what tier they receive; they may just want to attend the event at all costs, i.e. any tier. With one waiting list per tier, the only way to implement this would be to allow the user to be in multiple queues. It is fundamentally wrong to have a single user with two tickets, so this would introduce concurrency challenges to ensure they only receive one ticket, should they happen to be at the front of two waiting lists simultaneously.
    \item If an order is placed at the tier level, a user would have to place one order per tier, which is outrageous from a user-experience perspective, and makes it much harder to keep track of attendees and ensure no one receives two tickets.
\end{enumerate}

For these reasons, I decided instead to have one waiting list per fixture. Anyone who wanted sold out tickets would join this waiting list, and be able to indicate not only what tiers they were interested in, but also their order of preference, for example an individual may choose to accept tickets for either dress circle or stalls tickets, with a preference for the stalls if both become available at the same time.

The allocation algorithm starts by finding what tiers are available, and how many tickets in each. It then walks down the queue, querying each person's order. If it can satisfy the demand, it makes the individual an offer, before decrementing the number of tickets for the fulfilled tier. Most importantly, in order to remain fair, the algorithm must not be able to discover new tickets mid-execution (as people already passed over may have wanted those tickets); it instead has to run periodically.

I considered giving users an indication of their position in the waiting list, either an absolute number, or estimated time to the front, however this could encourage abuse. Although contrived, a tout knowingly at the front of the queue could demand money for dropping out if they knew someone was behind them. In fact, it is difficult to provide any sort of metric that is no exploitable. The queue must behave like a dark pool in trading.

\subsection{Refunds}

Payment is not taken until an offer of a ticket is accepted. There are two refund policies offered by the system:

\begin{description}
    \item[Always] Refunds are always accepted and processed immediately.
    \item[Reallocate] Refunds are only processed when the system can give the ticket to someone else, in order to protect the event organiser's revenue. New tickets will always be used to fulfil demand before returned ones, therefore refunds that fall into this policy will only be accepted if there is a waiting list for the tier.
\end{description}

I considered a third option, \textit{never}, however there is little reason for this over \textit{reallocate} from any stakeholder's perspective. It would not affect the event organiser, as they receive the revenue either way, and it would be worse for the performer, as someone would be forced to attend (or not attend) who did not want to go.

\subsection{Groups}

Now that single tickets were worked out, I attacked the issue of scaling this to groups, which meant revisiting all of the challenges I went through to handle individuals. This was a deliberate decision, as the leap from nothing to groups would have been somewhat overwhelming.

\begin{itemize}
    \item Groups need to be treated as an individual unit of allocation. They cannot complete checkout in isolation, as there would be no guarantee of everyone succeeding in receiving a ticket.
    \item There is the concept of gifts, where it is not always simple to indicate or represent who pays for who, which is compounded in the case of refunds.
    \item As per the requirements, at no point can one person own multiple tickets, however this is at odds with the concept of a single order.
\end{itemize}

I was glad to have a solid foundation in the form of single orders to adapt. I found it much easier to keep something working and expand on it rather than handle so many more variables.

First, I had to decide whether to impose a limit on the number of tickets that could be ordered as a group. Existing systems impose an upper bound to prevent abuse, however given ticket transfers are prohibited, there is no advantage to constraining the size of a group. It was noticeable how properly solving this core problem meant I was able to strip out many workarounds and duct tape like this that had been applied to other systems. Spending time on a few key decisions had saved a vast amount of complexity.

\subsubsection{Ordering Process}

% order individually
Next, I had to ensure all members of a group received tickets at the same time, or not at all. My first attempt involved people explicitly defining dependencies when making an order, for example person A specifies that they only want a ticket if persons B and C also get one; person B specifies they only want one if A and C receive one and so on. There were two key issues with this system:

\begin{enumerate}
    \item If everyone is forced to buy a ticket individually, they would have to co-ordinate seat selection and so on, all while others are doing the same. The chances of frustration are high, especially for popular events.
    \item It could be very complicated to calculate the dependencies when doing the allocation, particularly if the graph of members is not fully connected.
\end{enumerate}

% one on behalf of group
More simply, this method was just too complicated to ask users to understand, and would cause too many errors. I needed something more high-level and intuitive, and quickly realised the only sane way of doing this involved one person placing an order on behalf of the entire group. In the case of fixed seating, this would make it much easier to allocate a contiguous block of seating, and also creates a simple blob for the allocation system to fulfil or ignore. Having one person in charge also reduces confusion in case of issues, and also load on the system. There is no advantage to everyone in a group simultaneously trying to buy tickets to increase the likelihood of ``getting through''.

% now an issue of ownership; can associate with people at time of order
The issue with this was that of a single person owning many tickets. I needed a way to distinguish between someone genuinely purchasing tickets for friends, and someone who buys with the intention of touting. It took me far too long to realise that the key difference, and what is exploitable here, is that the person buying tickets for friends knows who those people are at the point of purchase. Whoever placed the order could enter friends' details, set who they wanted to pay for, and never be in control of any distribution.

% enter after offer received? no, before
To maintain the property of minimising the amount of information required before a user can join the waiting list, at first I decided to allow the group leader to enter names of attendees after an offer has been received. However, if a user had already purchased a ticket, this would mean it was not detected until after purchase, and secondly it would give touts a window to advertise that they had received tickets and solicit buyers. I realised it is more subtle and stricter than a genuine buyer knowing attendees at the point of purchase; to prevent the above, a buyer would have to specify who would attend before even knowing whether they had tickets. Therefore, in order to place an order, the group leader specifies a tier and a list of attendees, all of whom must already be registered with a validated photo. This is so the ticket cannot be touted and the buyer's photo added later. As part of the order processing workflow, a check would be made to ensure no users already had an active order for the fixture, to avoid the issue of multiple tickets per user.

\subsubsection{Waiting List}

Groups can share the current waiting list for individuals. Instead of allocating a single ticket, the algorithm simply passes over the group until a sufficient number are available. There are two potential issues here.

\begin{enumerate}
    \item If the amount of time between allocations is too short, not enought tickets will have accumulated to fulfil larger orders. A group of 8 at the front of the queue may never be served if the average rate of return is 8 tickets per minute, and the algorithm runs once every 30 seconds.
    \item It can be assumed that in the case of reserved seating, groups will want to sit together.
\end{enumerate}

To solve the first issue, the allocation algorithm could potentially analyse the rate of ticket returns, and determine whether it can wait long enough to fulfil a large order, e.g. if 1 ticket is returned every minute, it could wait 5 minutes to fulful an order of 5. There would have to be a threshold on this to ensure turnover in the queue. Certainly, it should never be the case that smaller groups are not able to attend because the allocator is causing a log jam to satisfy a large group.

The second issue is made more complicated by the fact that existing orders cannot be moved around. All that can be done here is for the allocator to group tickets where possible (e.g. seats C23 and C24 would be a pair). It would then sort the size of these groups in ascending order, and starting at the front of the queue, try to find the smallest block size that would satisfy the order. This algorithm could be adapted depending on the type of seating available. The clear downside is that it would require detailed and accurate plans of all venues with reserved seating.

\subsubsection{Payment}

The final issue is that of payment, which also implicates gifted tickets. As one person places the order, my initial idea was to have them pay for everyone for simplicity, then when a friend confirms the purchase, charge their payment card for the amount for their ticket, and do a partial refund for the buyer. Taking a step back, I realised this was again far too convoluted, and it is unreasonable to expect the buyer to be out of pocket, or even have enough money in their account for everyone if they only want a single ticket. There is also the issue of what happens if someone rejects their ticket. I definitely wanted to remove refunding from the normal payment workflow, as it incurs additional card processing fees.

To address all of these problems, I created a solution based on the idea of delegating responsibility for payment. A user wants to order $t$ tickets, of which $o$ are for other people, with $g$ of those gifted. $t - o$ will be $0$ if the group leader is not attending, or $1$ if they are. They will pay for $(t - o) + g$ tickets, with $o - g$ others paying for their tickets individually. Example scenarios include:

\begin{itemize}
    \item A user orders a ticket for himself.
    \begin{itemize}
        \item $t = 1, o = 0, g = 0$
        \item User pays for $(1 - 0) + 0 = 1$ ticket
    \end{itemize}
    \item A user orders a ticket for herself and 4 friends.
    \begin{itemize}
        \item $t = 5, o = 4, g = 0$
        \item User pays for $(5 - 4) + 0 = 1$ ticket
    \end{itemize}
    \item A user orders a ticket for himself and 4 friends, 2 as gifts.
    \begin{itemize}
        \item $t = 5, o = 4, g = 2$
        \item User pays for $(5 - 4) + 2 = 3$ tickets
    \end{itemize}
    \item A user orders 2 tickets for her children, both as gifts.
    \textit{N.B. she is not attending the event.}
    \begin{itemize}
        \item $t = 2, o = 2, g = 2$
        \item User pays for $(2 - 2) + 2 = 2$ tickets
    \end{itemize}
\end{itemize}

I intuitively saw several roles emerging here. An order consisted of a set of transactions, with each of those transactions paying for one or more tickets. The total number of transactions for an order will be $t - g + (1 - (t - o)) = o - g + 1$. A transaction's payee may not even be paying for his own ticket.

% TODO state diagram of entire system

%now enable appendix numbering format and include any appendices
\appendix
\begin{appendices}

%\chapter{Proposal}
%Here's my project proposal.

\chapter{API Reference}
\label{appendix:api_reference}

Methods, parameters and return values for the Equaliser API.
% TODO

\chapter{Source Archive}
\label{appendix:source_archive}

Here's what's included and how to run it.
% TODO

\end{appendices}

\cleardoublepage  % http://tex.stackexchange.com/a/23503
\addcontentsline{toc}{chapter}{Bibliography}
%\nocite{*} % can add back in to trivially increase page count
\printbibliography

\end{document}
